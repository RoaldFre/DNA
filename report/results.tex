\section{Results}

\subsection{Denaturation of dsDNA}

\paragraph{Knotts III.A.1: 5'-GCGTCATACAGTGC-3'}

\begin{figure}[h] \begin{minipage}{7cm}
          \scalebox{0.8}{
                        \nonstopmode
                        \input{images/data_knotts1/data_T270.dat.tex}
                        \errorstopmode
                        \rule[-0.5cm]{0cm}{0cm}}


 \end{minipage} \begin{minipage}{7cm} \input{images/data_knotts1/data_T300} \end{minipage}
\begin{minipage}{7cm}
\input{images/data_knotts1/data_T330} \end{minipage} \begin{minipage}{7cm}
               \scalebox{0.9}{
                        \nonstopmode
                        \input{images/data_knotts1/data_T360.dat.tex}
                        \errorstopmode
                        \rule[-0.5cm]{0cm}{0cm}}

\end{minipage}\begin{center}
\input{images/data_knotts1/data_T390}
\caption{Basepair configuration (black: open, white: closed) for the 5'-GCGTCATACAGTGC-3' configuration of a B-DNA double helix (Knotts III.A.1, \cite{knotts2007coarse}) for different temperatures. Top-left: 270K; top-right: 300K; middle-left: 330K; middle-right: 360K; bottom: 390K. The timescale on the x-axis is always in seconds (about 30 nanoseconds simulated).}\end{center}
\end{figure}

\begin{figure}\begin{center}
-e                \scalebox{0.8}{
                        \nonstopmode
                        % GNUPLOT: LaTeX picture with Postscript
\begingroup
  \makeatletter
  \providecommand\color[2][]{%
    \GenericError{(gnuplot) \space\space\space\@spaces}{%
      Package color not loaded in conjunction with
      terminal option `colourtext'%
    }{See the gnuplot documentation for explanation.%
    }{Either use 'blacktext' in gnuplot or load the package
      color.sty in LaTeX.}%
    \renewcommand\color[2][]{}%
  }%
  \providecommand\includegraphics[2][]{%
    \GenericError{(gnuplot) \space\space\space\@spaces}{%
      Package graphicx or graphics not loaded%
    }{See the gnuplot documentation for explanation.%
    }{The gnuplot epslatex terminal needs graphicx.sty or graphics.sty.}%
    \renewcommand\includegraphics[2][]{}%
  }%
  \providecommand\rotatebox[2]{#2}%
  \@ifundefined{ifGPcolor}{%
    \newif\ifGPcolor
    \GPcolortrue
  }{}%
  \@ifundefined{ifGPblacktext}{%
    \newif\ifGPblacktext
    \GPblacktexttrue
  }{}%
  % define a \g@addto@macro without @ in the name:
  \let\gplgaddtomacro\g@addto@macro
  % define empty templates for all commands taking text:
  \gdef\gplbacktext{}%
  \gdef\gplfronttext{}%
  \makeatother
  \ifGPblacktext
    % no textcolor at all
    \def\colorrgb#1{}%
    \def\colorgray#1{}%
  \else
    % gray or color?
    \ifGPcolor
      \def\colorrgb#1{\color[rgb]{#1}}%
      \def\colorgray#1{\color[gray]{#1}}%
      \expandafter\def\csname LTw\endcsname{\color{white}}%
      \expandafter\def\csname LTb\endcsname{\color{black}}%
      \expandafter\def\csname LTa\endcsname{\color{black}}%
      \expandafter\def\csname LT0\endcsname{\color[rgb]{1,0,0}}%
      \expandafter\def\csname LT1\endcsname{\color[rgb]{0,1,0}}%
      \expandafter\def\csname LT2\endcsname{\color[rgb]{0,0,1}}%
      \expandafter\def\csname LT3\endcsname{\color[rgb]{1,0,1}}%
      \expandafter\def\csname LT4\endcsname{\color[rgb]{0,1,1}}%
      \expandafter\def\csname LT5\endcsname{\color[rgb]{1,1,0}}%
      \expandafter\def\csname LT6\endcsname{\color[rgb]{0,0,0}}%
      \expandafter\def\csname LT7\endcsname{\color[rgb]{1,0.3,0}}%
      \expandafter\def\csname LT8\endcsname{\color[rgb]{0.5,0.5,0.5}}%
    \else
      % gray
      \def\colorrgb#1{\color{black}}%
      \def\colorgray#1{\color[gray]{#1}}%
      \expandafter\def\csname LTw\endcsname{\color{white}}%
      \expandafter\def\csname LTb\endcsname{\color{black}}%
      \expandafter\def\csname LTa\endcsname{\color{black}}%
      \expandafter\def\csname LT0\endcsname{\color{black}}%
      \expandafter\def\csname LT1\endcsname{\color{black}}%
      \expandafter\def\csname LT2\endcsname{\color{black}}%
      \expandafter\def\csname LT3\endcsname{\color{black}}%
      \expandafter\def\csname LT4\endcsname{\color{black}}%
      \expandafter\def\csname LT5\endcsname{\color{black}}%
      \expandafter\def\csname LT6\endcsname{\color{black}}%
      \expandafter\def\csname LT7\endcsname{\color{black}}%
      \expandafter\def\csname LT8\endcsname{\color{black}}%
    \fi
  \fi
  \setlength{\unitlength}{0.0500bp}%
  \begin{picture}(9600.00,7680.00)%
    \gplgaddtomacro\gplbacktext{%
      \colorrgb{0.00,0.00,0.00}%
      \put(1116,845){\makebox(0,0)[r]{\strut{}0}}%
      \colorrgb{0.00,0.00,0.00}%
      \put(1116,1679){\makebox(0,0)[r]{\strut{}2}}%
      \colorrgb{0.00,0.00,0.00}%
      \put(1116,2514){\makebox(0,0)[r]{\strut{}4}}%
      \colorrgb{0.00,0.00,0.00}%
      \put(1116,3348){\makebox(0,0)[r]{\strut{}6}}%
      \colorrgb{0.00,0.00,0.00}%
      \put(1116,4183){\makebox(0,0)[r]{\strut{}8}}%
      \colorrgb{0.00,0.00,0.00}%
      \put(1116,5017){\makebox(0,0)[r]{\strut{}10}}%
      \colorrgb{0.00,0.00,0.00}%
      \put(1116,5851){\makebox(0,0)[r]{\strut{}12}}%
      \colorrgb{0.00,0.00,0.00}%
      \put(1116,6686){\makebox(0,0)[r]{\strut{}14}}%
      \colorrgb{0.00,0.00,0.00}%
      \put(1248,625){\makebox(0,0){\strut{}0}}%
      \colorrgb{0.00,0.00,0.00}%
      \put(2178,625){\makebox(0,0){\strut{}1e-08}}%
      \colorrgb{0.00,0.00,0.00}%
      \put(3108,625){\makebox(0,0){\strut{}2e-08}}%
      \colorrgb{0.00,0.00,0.00}%
      \put(4038,625){\makebox(0,0){\strut{}3e-08}}%
      \colorrgb{0.00,0.00,0.00}%
      \put(4968,625){\makebox(0,0){\strut{}4e-08}}%
      \colorrgb{0.00,0.00,0.00}%
      \put(5898,625){\makebox(0,0){\strut{}5e-08}}%
      \colorrgb{0.00,0.00,0.00}%
      \put(6828,625){\makebox(0,0){\strut{}6e-08}}%
      \colorrgb{0.00,0.00,0.00}%
      \put(7758,625){\makebox(0,0){\strut{}7e-08}}%
      \colorrgb{0.00,0.00,0.00}%
      \put(8688,625){\makebox(0,0){\strut{}8e-08}}%
      \colorrgb{0.00,0.00,0.00}%
      \put(610,3974){\rotatebox{90}{\makebox(0,0){\strut{}\rule{0pt}{-1.5cm}Number of closed basepairs}}}%
      \colorrgb{0.00,0.00,0.00}%
      \put(4968,295){\makebox(0,0){\strut{}Time (seconds)}}%
    }%
    \gplgaddtomacro\gplfronttext{%
      \csname LTb\endcsname%
      \put(1908,1898){\makebox(0,0)[r]{\strut{}270K}}%
      \csname LTb\endcsname%
      \put(1908,1678){\makebox(0,0)[r]{\strut{}300K}}%
      \csname LTb\endcsname%
      \put(1908,1458){\makebox(0,0)[r]{\strut{}330K}}%
      \csname LTb\endcsname%
      \put(1908,1238){\makebox(0,0)[r]{\strut{}360K}}%
      \csname LTb\endcsname%
      \put(1908,1018){\makebox(0,0)[r]{\strut{}390K}}%
    }%
    \gplbacktext
    \put(0,0){\includegraphics{images/knotts1_lines}}%
    \gplfronttext
  \end{picture}%
\endgroup

                        \errorstopmode
                        \rule[-0.5cm]{0cm}{0cm}}

\caption{Longer (80 nanoseconds) time-evolution of the denaturation (number of closed basepairs) of the Knotts III.A.1 B-DNA configuration.}\end{center}\end{figure}

\paragraph{Knotts III.A.3: L60B36 Bubble Formation} This is a 60bp strand of dsDNA studied by, among others, Zeng \etal \cite{zeng}. The base sequence reads 5'-CCGCCAGCGG CGTTATTACATTTAATTCTTAAGTATTATAAGTAATATGGCCGCTGCGCC-3' and is designed to form a `bubble' at temperatures around the melting point because of the weaker base pairing for the Ab and Tb bases; with the stronger Cb and Gb bondings at the extrema of the strand. 

\begin{figure}[h] \begin{minipage}{7cm}

               \scalebox{0.9}{
                        \nonstopmode
                        \input{images/data_L60B36_70ns/L60B36_70ns_T340.dat.tex}
                        \errorstopmode
                        \rule[-0.5cm]{0cm}{0cm}}

 \end{minipage} \begin{minipage}{7cm} -e \begin{figure}[htb]
       \begin{center}
               \scalebox{0.9}{
                        \nonstopmode
                        \input{images/data_L60B36_70ns/L60B36_70ns_T350.dat.tex}
                        \errorstopmode
                        \rule[-0.5cm]{0cm}{0cm}}
                \caption{}
        \end{center}
\end{figure}

 \end{minipage}
\begin{minipage}{8cm}

               \scalebox{0.9}{
                        \nonstopmode
                        % GNUPLOT: LaTeX picture with Postscript
\begingroup
  \makeatletter
  \providecommand\color[2][]{%
    \GenericError{(gnuplot) \space\space\space\@spaces}{%
      Package color not loaded in conjunction with
      terminal option `colourtext'%
    }{See the gnuplot documentation for explanation.%
    }{Either use 'blacktext' in gnuplot or load the package
      color.sty in LaTeX.}%
    \renewcommand\color[2][]{}%
  }%
  \providecommand\includegraphics[2][]{%
    \GenericError{(gnuplot) \space\space\space\@spaces}{%
      Package graphicx or graphics not loaded%
    }{See the gnuplot documentation for explanation.%
    }{The gnuplot epslatex terminal needs graphicx.sty or graphics.sty.}%
    \renewcommand\includegraphics[2][]{}%
  }%
  \providecommand\rotatebox[2]{#2}%
  \@ifundefined{ifGPcolor}{%
    \newif\ifGPcolor
    \GPcolortrue
  }{}%
  \@ifundefined{ifGPblacktext}{%
    \newif\ifGPblacktext
    \GPblacktexttrue
  }{}%
  % define a \g@addto@macro without @ in the name:
  \let\gplgaddtomacro\g@addto@macro
  % define empty templates for all commands taking text:
  \gdef\gplbacktext{}%
  \gdef\gplfronttext{}%
  \makeatother
  \ifGPblacktext
    % no textcolor at all
    \def\colorrgb#1{}%
    \def\colorgray#1{}%
  \else
    % gray or color?
    \ifGPcolor
      \def\colorrgb#1{\color[rgb]{#1}}%
      \def\colorgray#1{\color[gray]{#1}}%
      \expandafter\def\csname LTw\endcsname{\color{white}}%
      \expandafter\def\csname LTb\endcsname{\color{black}}%
      \expandafter\def\csname LTa\endcsname{\color{black}}%
      \expandafter\def\csname LT0\endcsname{\color[rgb]{1,0,0}}%
      \expandafter\def\csname LT1\endcsname{\color[rgb]{0,1,0}}%
      \expandafter\def\csname LT2\endcsname{\color[rgb]{0,0,1}}%
      \expandafter\def\csname LT3\endcsname{\color[rgb]{1,0,1}}%
      \expandafter\def\csname LT4\endcsname{\color[rgb]{0,1,1}}%
      \expandafter\def\csname LT5\endcsname{\color[rgb]{1,1,0}}%
      \expandafter\def\csname LT6\endcsname{\color[rgb]{0,0,0}}%
      \expandafter\def\csname LT7\endcsname{\color[rgb]{1,0.3,0}}%
      \expandafter\def\csname LT8\endcsname{\color[rgb]{0.5,0.5,0.5}}%
    \else
      % gray
      \def\colorrgb#1{\color{black}}%
      \def\colorgray#1{\color[gray]{#1}}%
      \expandafter\def\csname LTw\endcsname{\color{white}}%
      \expandafter\def\csname LTb\endcsname{\color{black}}%
      \expandafter\def\csname LTa\endcsname{\color{black}}%
      \expandafter\def\csname LT0\endcsname{\color{black}}%
      \expandafter\def\csname LT1\endcsname{\color{black}}%
      \expandafter\def\csname LT2\endcsname{\color{black}}%
      \expandafter\def\csname LT3\endcsname{\color{black}}%
      \expandafter\def\csname LT4\endcsname{\color{black}}%
      \expandafter\def\csname LT5\endcsname{\color{black}}%
      \expandafter\def\csname LT6\endcsname{\color{black}}%
      \expandafter\def\csname LT7\endcsname{\color{black}}%
      \expandafter\def\csname LT8\endcsname{\color{black}}%
    \fi
  \fi
  \setlength{\unitlength}{0.0500bp}%
  \begin{picture}(5760.00,3840.00)%
    \gplgaddtomacro\gplbacktext{%
      \colorrgb{0.00,0.00,0.00}%
      \put(110,1986){\rotatebox{90}{\makebox(0,0){\strut{}\rule{0pt}{-1.5cm}Basepair configuration}}}%
      \colorrgb{0.00,0.00,0.00}%
      \put(2980,-128){\makebox(0,0){\strut{}Time (seconds)}}%
    }%
    \gplgaddtomacro\gplfronttext{%
      \colorrgb{0.00,0.00,0.00}%
      \put(616,3074){\makebox(0,0)[r]{\strut{}10}}%
      \colorrgb{0.00,0.00,0.00}%
      \put(616,2543){\makebox(0,0)[r]{\strut{}20}}%
      \colorrgb{0.00,0.00,0.00}%
      \put(616,2013){\makebox(0,0)[r]{\strut{}30}}%
      \colorrgb{0.00,0.00,0.00}%
      \put(616,1483){\makebox(0,0)[r]{\strut{}40}}%
      \colorrgb{0.00,0.00,0.00}%
      \put(616,952){\makebox(0,0)[r]{\strut{}50}}%
      \colorrgb{0.00,0.00,0.00}%
      \put(616,422){\makebox(0,0)[r]{\strut{}60}}%
      \colorrgb{0.00,0.00,0.00}%
      \put(748,202){\makebox(0,0){\strut{}0}}%
      \colorrgb{0.00,0.00,0.00}%
      \put(1385,202){\makebox(0,0){\strut{}1e-08}}%
      \colorrgb{0.00,0.00,0.00}%
      \put(2022,202){\makebox(0,0){\strut{}2e-08}}%
      \colorrgb{0.00,0.00,0.00}%
      \put(2658,202){\makebox(0,0){\strut{}3e-08}}%
      \colorrgb{0.00,0.00,0.00}%
      \put(3295,202){\makebox(0,0){\strut{}4e-08}}%
      \colorrgb{0.00,0.00,0.00}%
      \put(3932,202){\makebox(0,0){\strut{}5e-08}}%
      \colorrgb{0.00,0.00,0.00}%
      \put(4569,202){\makebox(0,0){\strut{}6e-08}}%
      \colorrgb{0.00,0.00,0.00}%
      \put(5206,202){\makebox(0,0){\strut{}7e-08}}%
    }%
    \gplbacktext
    \put(0,0){\includegraphics{images/data_L60B36_70ns/L60B36_70ns_T360}}%
    \gplfronttext
  \end{picture}%
\endgroup

                        \errorstopmode
                        \rule[-0.5cm]{0cm}{0cm}}


 \end{minipage} \begin{minipage}{6cm}
\includegraphics[width=7cm]{images/L60B36_bubble2.png}\\
\includegraphics[width=7cm]{images/L60B36_bubble1.png} \end{minipage}
\begin{center}
\caption{Basepairing configuration (y-axis) for bubble formation in the L60B36 dsDNA configuration (white: closed basepairs, black: open basepairs) for a time-evolution of 80 nanoseconds (x-axis) at 340K (top-left), 350K (top-right) and 360K (bottom-left). The process of bubble formation is illustrated in two bubble configurations at 340K (bottom-right panel, top configuration) and at 350K (bottom configuration). }
\end{center}
\end{figure}





