\section{Conclusions}

Time scaling behavior of hairpin formation by single stranded DNA structures was simulated with a coarse grained model modeling the DNA structure with three sites (base, sugar, phosphate) per nucleotide.

The melting temperatures of DNA hairpins we found seem to comply reasonably with experimental values by Vallone \etal\ \cite{vallone1999melting}. The small deviations that were observed can potentiallly be attributed to the omission of hydrophobic interactions in the model, which are nonetheless shown to play a significant role for small hairpins \cite{kuznetsov2001semiflexible}.

We found that the time scale of zipping of DNA hairpins behaves like
\begin{equation}
\tau_\text{zipping} \sim N^{\betazip} \qquad \qquad \text{with}\ \betazip = 1.33\pm0.14
\end{equation}
in function of the stem length $N$. Other authors \cite{carlon2011anomalous}, using more simple Monte Carlo methods, found time scaling with exponent $\betazip = 1.37 \pm 0.02$, which is well within the error margin of our result. Note that the model in \cite{carlon2011anomalous} did not take into account the helical nature of DNA. Thus, as far as the scaling behaviour is concerned, the winding of a DNA hairpin during formation plays no significant role.

The unzipping of DNA hairpins in our model is found to have a timescale yielding
\begin{equation}
\tauunzip \sim N^{\betazip} \qquad \qquad \text{with}\ \betaunzip = 2.57\pm0.10.
\end{equation}
Previous simulations by Baiesi, Barkema, Carlon \& Panja \cite{carlon2010unwinding} and Baiesi \& Livi \cite{baiesi2009multiple} with simplified models that take into account the helical nature of DNA yielded values $\betaunzip = 2.57(3)$ and $\betaunzip = 3$, respectively. Our simulations thus confirm the scaling law with exponent $\betaunzip \approx 2.6$ of the first model \cite{carlon2010unwinding} for the case of hairpin unzipping. Note that the simple model in \cite{carlon2011anomalous} that successfully predicted the hairpin zipping time above, does \emph{not} predict the correct unzipping scaling law --- it predicts $\betaunzip = 1$. Hence, far as DNA hairpin \emph{un}zipping is concerned, the helical structure of DNA plays a crucial role.


Lastly, improvements to our methods certainly can be made. All simulations presented in this paper were done in the time frame of less than one week\footnote{As was mentioned, the majority of the time was spent implementing the algorithms and verifying the absolute correctness of the code before any meaningful simulation could be run.}
--- it is inevitable that longer and more numerous simulations will yield better and more significant results (certainly concerning the error margins on the time scaling behavior that we found). Furthermore, we think that the model we implemented (although fast already) and the simulation scripts we used can be optimized further to attain more efficient CPU time.

