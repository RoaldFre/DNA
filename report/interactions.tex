\subsection{Interactions}

Interactions between the sites are modelled through several potential energies.
These potentials are taken taken mainly from the potential terms as defined by Florescu \& Joyeux \cite{florescu2011thermal}, and are an improvement over the original terms by Knotts \etal \cite{knotts2007coarse}. Deviations from \cite{florescu2011thermal} will be explicitly mentioned.
TODO CHECK THAT WE MENTION IT :P

We will now sketch the specific potential energy terms, one by one.

\begin{table}[hbt]
\label{couplingConstants}
\begin{center}
\caption{Coupling constants for the various potentials. The basic energy unit $\varepsilon$ is equal to $0.26$\,kcal/mol, or $1.81 \times 10^{-21}$\,J.}

\begin{tabular}{cc||cc}
Parameter & Value & Parameter & Value\\\hline
$k_1$ & $\varepsilon$ per $\Angstrom^2$ &
	$k_\phi$ & $4\varepsilon$ \\
$k_2$ & $100\varepsilon$ per $\Angstrom^2$ &
	$\varepsilon_\text{GC}$ & $4\varepsilon$\\
$k_\theta$ & $400\varepsilon$ per (radian)$^2$ &
	$\varepsilon_\text{AT}$ & $\frac{2}{3} \varepsilon_\text{GC}$ \\
\end{tabular}
\end{center}
\end{table}


\paragraph{Bond potential}
The intramolecular bonds between connected sites in the same DNA strand is modelled by the potential
\begin{equation}
\Vbond
= \sum_{\substack{\text{bound sites}\\i}} \left[
	k_1 \left(d_i - d_{0_i}\right)^2
	+ k_2 \left(d_i - d_{0_i}\right)^4
\right],
\end{equation}
where $d_i$ is the distance between the sites that constitute the $i$th bond and $d_{0_i}$ is the equilibrium distances for that bonds, as determined by the standard B-form structure of double strand DNA. The explicit values of these equilibrium distances and of the coupling constants $k_1$ and $k_2$ can be found it table \ref{geometricConstants} and \ref{couplingConstants}, respectively.


\paragraph{Bond angle potential}
The angle $\theta$ that forms between three consecutively bound sites in a DNA strand is regulated by the harmonic potential
\begin{equation}
\Vang
= \sum_{\substack{\text{angle triplets}\\i}}
	\frac{k_\theta}{2} \left(
		\theta_i - \theta_{0_i}
	\right)^2,
\end{equation}
where the equilibrium angle $\theta_{0_i}$ is defined from the DNA B-form (values are tabulated in table \ref{geometricConstants}). The coupling constant $k_\theta$ is defined in table \ref{couplingConstants}.


\paragraph{Dihedral angle potential}
This potential regulates the dihedral angle $\phi$ between four consecutive bound sites on the same strand. It is given by
\begin{equation}
\Vdih
= \sum_{\substack{\text{dihedrals}\\i}}
	k_\phi \left[ 1 - \cos (\phi_i - \phi_{0_i}) \right],
\end{equation}
with $\phi_{0_i}$ the equilibrium dihedral angle from the DNA B-form definitions. The constants are defined in tables \ref{geometricConstants} and \ref{couplingConstants}.


\paragraph{Stacking potential}
The stacking potential is a Lennard-Jones type potential describing the intra-strand base stacking phenomena. It helps regulate the rigidity of the DNA backbone:
\begin{equation}
\Vstck
= \sum_{\substack{\text{stack pairs}\\i}}
	\varepsilon \left[
       		   \left(\frac{\dstck_i}{r_i} \right)^{12}
       		- 2\left(\frac{\dstck_i}{r_i} \right)^{6}
       	\right],
\end{equation}
where $r_i$ is the distance between the bases of the stacking pair. The sum over stack pairs runs between the $i$th and $(i+1)$th base, but also between the $i$th and $(i+2)$th base of the same strand.
This corresponds to an (off latice) G$\bar{\text o}$-type native contact scheme\cite{hoangcieplak, cieplak2003folding} with a cut-off distance of $9$\,\Angstrom.

The equilibrium distances $d_{ij}^\text{stck}$ are determined from the B isoform data in table \ref{dnaStructureData}.


\paragraph{Base pairing potential}
A hydrogen bonding interaction between two complimentary bases is modelled by
\begin{equation}
\Vbp
= \sum_{\substack{\text{base pairs}\\i}}
\varepsilon^\text{bp}_i \left[
	  5\left(\frac{\dbp_i}{r_i} \right)^{12}
	- 6\left(\frac{\dbp_i}{r_i} \right)^{10}
\right]
\end{equation}
where the equilibrium distances $d^\text{bp}_i$ and coupling constants $\varepsilon^\text{bp}_i$ depend on the type base pair. The explicit values are given in table \ref{couplingConstants}. No interaction is assumed between non-complementary base pairs.


\paragraph{Coulombic potential}
The final potential in our model is the screened electrostatic Coulomb interaction between the charged phosphate groups situated on the backbone of the DNA strands. It is modeled using a Debye-H\"uckel approximation with a Debye length $\kappa_D$ depending on the salt concentration of the environment,
\begin{equation}
\Vqq
= \sum_{\substack{\text{phosphate pairs}\\i}}
	\frac{e^2}{4 \pi \varepsilon_0 \varepsilon_k r_i}
       		\exp \left(- \frac{r_i}{k_D} \right),
\label{Vcoulomb}
\end{equation}
where the Debye length can be written as
\begin{equation}
\kappa_D
= \sqrt{ \frac{\varepsilon_0 \varepsilon_k RT}{2N^2_A e^2 I}}
\end{equation}
where we use the vacuum permittivity $\varepsilon_0$, Avogadro's number $N_A$, the elementary electric charge $e$ and the ionic strength $I$. For a realistic value of the ionic strength $[\text{Na}^+] = 50$\,mM (milimolair, equal to milimol/liter) this yields $\kappa_D$ = 13.603\,\Angstrom. The remaining constant in \eqref{Vcoulomb} is the dielectric constant $\varepsilon_k = 80$ for water at room temperature.


\paragraph{Exclusion potential}
Steric repulsion between sites is modelled using the repulsive part of a Lennard-Jones potential
\begin{equation}
\Vexcl
= \sum_{\substack{\text{exclusion pairs}\\i}}
\begin{cases}
	\varepsilon \left[
		   \left(\dfrac{\dexcl_i}{r_i} \right)^{12}
		- 2\left(\dfrac{\dexcl_i}{r_i} \right)^{6}
       	\right] + \varepsilon
	\qquad &\text{if }r_i < \dexcl_i,\\
	0
	\qquad &\text{if }r_i \geq \dexcl_i.
\end{cases}
\end{equation}
A pair of sites (not necessarily on the same strand) is considered an `exclusion pair' if it does not form a bond (in the sense of contributing to \Vbond).
The cut off distances $\dexcl_i$ are set to $1\,\Angstrom$ for exclusion between base pairs, and $5.5\,\Angstrom$ between all other base pairs.

In the original and follow-up versions of the 3SPN model (Knotts \etal \cite{knotts2007coarse}, Sambriski \etal \cite{sambriski2009mesoscale}), the cut off distance between non-base pairs was set to the higher value of $6.86\,\Angstrom$. However, this value causes some artificial swelling of the DNA strand beyond its equilibrium B form. We therefore opted to lower this exclusion cut off, effectively reducing the steric radius of the sites. For a value of $5.5\,\Angstrom$, the equilibrium distances between the sites are fully determined by the equilibrium constants in \Vbond and there is no swelling.

Note that Florescu \& Joyeux \cite{florescu2011thermal} reported that the exclusion potential does not play a significant role when simulating melting temperatures and renaturation properties of long double stranded DNA helices. They opted to leave it out altogether.

However, we are interested in the scaling behaviour of DNA hairpin zipping time and, as mentioned in the introduction, the helical nature of DNA might play a crucial role in that case \cite{carlon2010unwinding}.
In this spirit, we held on to the exclusion interaction. Note that we verified that our reduced steric cut off distance of $5.5\,\Angstrom$ is still sufficient to forbid overlap between (and passage through) DNA strands. Hence, the strands will indeed be forced to rotate during (de)naturation.


\paragraph{Total potential}
The total interaction is not merely a sum of the individual terms.
Following Knotts \etal \cite{knotts2007coarse}, two sites are excluded from non-bonded interactions (\Vstck, \Vbp, \Vqq, \Vexcl) if they constitute a bond (as per \Vbond).

On top of that, \Vbp, \Vqq and \Vexcl are modeled as mutually exclusive, where the exclusion force has the lowest precedence (note that base pairing and Coulomb repulsion cannot happen simultaneously between the same two sites).
Indeed, the exclusion force merely acts to keep two sites from passing through each other, something that is already achieved by the base pairing and Coulomb repulsion.

Because non-bonded potentials are short ranged, we can truncate them at a sufficiently large distance.
This enables us to work with a space partitioning acceleration scheme discussed below.
We chose to truncate the potentials at a distance of $20\,\Angstrom$. After truncation, the potentials are shifted so they yield a potential energy of zero when calculated at (and beyond) the truncation distance.



\begin{table}[htb]
\label{geometricConstants}
\caption{Geometric constants for the potential energy functions as defined above. Important to note is that a phosphate can bind to a sugar at the 5' or the 3' carbon. The table is reproduced from Knotts \etal \cite{knotts2007coarse}, with our own value for \dexcl.}
\begin{center}
\begin{tabular}{cc@{\qquad}cc}
\hline
Bond& $d_0$ (\Angstrom) & Bond Angle&     $\theta_0$ (degree) \\\hline
S(5')--P & 3.899 &        S(5')--P--(3')S & 94.49\\
S(3')--P & 3.559 &        P--(5')S(3')--P & 120.15 \\
S--Ab    & 6.430 &        P--(5')S--Ab    & 113.13\\
S--Tb    & 4.880 &        P--(3')S--Ab    & 108.38\\
S--Cb    & 4.921 &        P--(5')S--Tb    & 102.79\\
S--Gb    & 6.392 &        P--(3')S--Tb    & 112.72\\
&&                        P--(5')S--Cb    & 103.49\\
&&                        P--(3')S--Cb    & 112.39\\
&&                        P--(5')S--Gb    & 113.52\\
&&                        P--(3')S--Gb    & 108.12\\
& & & \\
\hline
Dihedral Angle & $\phi_0$ (degree) & Nonbonded & Length (\Angstrom) \\
\hline
P-(5')S(3')-P-(5')S & $-154.80$&   $\dstck$ & Derived from table \ref{geometricConstants} \\
S(3')-P-(5')S(3')-P & $-179.17$&   & \\  
Ab-S(3')-P-(5')S &    $ -22.60$&   $\dbp_\text{AT}$ & $2.9002$ \\
S(3')-P-(5')S-Ab &    $  50.69$&   $\dbp_\text{GC}$ & $2.8694$ \\ 
Tb-S(3')-P-(5')S &    $ -33.42$&   & \\
S(3')-P-(5')S-Tb &    $  54.69$&   $\dexcl$         & $1.0$ (mismatched bases) \\
Cb-S(3')-P-(5')S &    $ -32.72$&   $\dexcl$         & $5.5$ (otherwise) \\
S(3')-P-(5')S-Cb &    $  54.50$&   & \\
Gb-S(3')-P-(5')S &    $ -22.30$&   & \\
S(3')-P-(5')S-Gb &    $  50.66$&   & \\
\hline
\end{tabular}
\end{center}
\end{table}

